%----------------------------------------------------------------------------------------
%   DESCRIPTION OF THE ASSIGNMENT OR REPORT
%----------------------------------------------------------------------------------------
% BEGIN_FOLD
% !!! WARNING COPY WITH mcode FILE !!!
% This title page is capable of being compiled as is. This is not useful for 
% including it in another document. To do this, you have two options: 
%
% 1) Copy/paste everything between \begin{document} and \end{document} 
% starting at \begin{titlepage} and paste this into another LaTeX file where you 
% want your title page.
% OR
% 2) Remove everything outside the \begin{titlepage} and \end{titlepage} and 
% move this file to the same directory as the LaTeX file you wish to add it to. 
% Then add \input{./title_page_1.tex} to your LaTeX file where you want your
% title page.
%
%%%%%%%%%%%%%%%%%%%%%%%%%%%%%%%%%%%%%%%%%
%\title{Title page with logo}
%END_FOLD
%----------------------------------------------------------------------------------------
%   PACKAGES AND OTHER DOCUMENT CONFIGURATIONS
%----------------------------------------------------------------------------------------


\documentclass[12pt]{article}
\usepackage[english]{babel}
\usepackage[utf8x]{inputenc} %encoding
\usepackage[toc,page]{appendix} %Appendix handling 
%\usepackage[a4paper, margin=0.7in]{geometry} if you want to set margins
\usepackage{geometry} %for changing margins for selected pages only
\usepackage{setspace} %for changing line spacing for selected paragraphs use: \begin{spacing}{1.5} *TEXT* \end{spacing}
\usepackage{amssymb} %for more symbols
\usepackage{graphicx} 
\usepackage[colorinlistoftodos]{todonotes} %side notes
\usepackage{float}%float objects around
				 % for text sub/superscripts: \textsuperscript{...} or  \textsubscript{...}
\usepackage{booktabs} %For tables imported from Excel
\usepackage{multirow} %For tables imported from Excel
\usepackage{colortbl} %For color tables imported from Excel
\usepackage{xcolor} %For color tables imported from Excel
\usepackage{rotating} %For color tables imported from Excel
\usepackage[nottoc]{tocbibind} %For bibliography listing in ToC

%MATLAB Listings
\usepackage[numbered, framed]{mcode} %For Matlab code listings
\lstset{stepnumber  = 5, % Line numbers go in steps of 4
	breaklines  = true,
	breakautoindent=true, 
	breakindent=10pt,
	breakatwhitespace   = false,
	prebreak= \space,
	postbreak   = \space}

% Usage: 
%in-text listing: \mcode{}
%Frame listing: environment \begin{lstlisting}[caption=Example caption\label{tag}] \end{lstlisting}

%From file: \lstinputlisting{/path_to_mfile/my_mfile.m}
%list of listings: \lstlistoflistings

% NOMENCLATURE This code creates the groups for NOMENCLATURE
% ----------------------------------------------------------------------------
\usepackage[intoc]{nomencl}% for symbol nomenclature
\makenomenclature

\usepackage{etoolbox}
\renewcommand\nomgroup[1]{%
	\item[\bfseries
	\ifstrequal{#1}{C}{Physics Constants}{%
	\ifstrequal{#1}{A}{Abbreviations}{%
	\ifstrequal{#1}{G}{Greek Symbols}{%
	\ifstrequal{#1}{S}{Roman Symbols}{}}}}%
	]}

% This will add the units
%----------------------------------------------
\newcommand{\nomunit}[1]{%
	\renewcommand{\nomentryend}{\hspace*{\fill}#1}}
%----------------------------------------------

%example in-text commands for the nomenclature:

%\nomenclature[C]{$h$}{Plank Constant
%	\nomunit{$6.62607 \times 10^{-34}\, Js$}}
%%


% HYPERLINKS AND PDF METADATA

\usepackage{color}   %May be necessary if you want to color links
\usepackage[
pdftex,
colorlinks=false, %set true if you want colored links
linktoc=all,     %set to all if you want both sections and subsections linked
linkcolor=blue,  %choose some color if you want links to stand out
hidelinks,		%to make the cross references invisible (red frames)
]{hyperref}
\hypersetup{
	pdfauthor={Jakub Włodarczyk},
	pdftitle={Master Thesis},
	%% pdfsubject={The Subject},
	%% pdfkeywords={Some Keywords},
	pdfproducer={Latex},
	pdfcreator={pdflatex}
}
\usepackage{amsmath} %math script handling
\usepackage{cleveref} %for full name of the reference as a hyperlink, e.g. "Figure 3" instead of just "3". Use \cref{label} to get the reference


%% more info on floats and figues:  https://en.wikibooks.org/wiki/LaTeX/Floats,_Figures_and_Captions

\begin{document}

\begin{titlepage}
\newgeometry{left=1cm, right=1cm}
\newcommand{\HRule}{\rule{\linewidth}{0.5mm}} % Defines a new command for the horizontal lines, change thickness here

\center % Center everything on the page
\numberwithin{equation}{section} %to number equations with respect to the section number
 
%----------------------------------------------------------------------------------------
%   HEADING SECTIONS
%----------------------------------------------------------------------------------------

\textsc{\LARGE KTH Royal Institute of Technology}\\[1cm] % Name of your university/college
\begin{spacing}{1.5}
\textsc{\LARGE ZHAW Z\"urich University of Applied Sciences}\\[0.5cm] % Name of your university/college
\end{spacing}
\textsc{\Large Institute of Computational Physics}\\[0.5cm] % Major heading such as course name
%\textsc{\large Minor Heading}\\[0.5cm] % Minor heading such as course title



%----------------------------------------------------------------------------------------
%   TITLE SECTION
%----------------------------------------------------------------------------------------

\HRule \\[0.4cm]
{ \Huge  \bfseries Master Thesis }\\[0.1cm] % Title of your document
\HRule \\[1cm]
 
%----------------------------------------------------------------------------------------
%   AUTHOR SECTION
%----------------------------------------------------------------------------------------

\begin{minipage}[b]{0.3\textwidth}
	\large
	\emph{Author:}\\
	Jakub \textsc{Włodarczyk} % Your name
\end{minipage}
\begin{minipage}[b]{0.35\textwidth}
	\large
	\emph{Supervisor ZHAW:} \\
	Prof. J\"urgen \textsc{Schumacher} % Supervisor's Name
\end{minipage}
\begin{minipage}[b]{0.3\textwidth}
	\large
	\emph{Supervisor KTH:} \\
	Prof. G\"oran \textsc{Lindbergh} % Supervisor's Name
\end{minipage}\\[2cm]

% If you don't want a supervisor, uncomment the two lines below and remove the section above
%\Large \emph{Author:}\\
%\textsc{jakub włodarczyk}\\
%\small 930907-1455 \quad jakubw@kth.se\\[1.5cm]
%
%----------------------------------------------------------------------------------------
%   DATE SECTION
%----------------------------------------------------------------------------------------
\large Winterthur, Switzerland\\
{\large \today}\\[2cm] % Date, change the \today to a set date if you want to be precise

%----------------------------------------------------------------------------------------
%   LOGO SECTION
%----------------------------------------------------------------------------------------
\begin{center}
\begin{minipage}{0.4\textwidth}
\includegraphics[width=4.3cm]{Figures/_Frontpage/Kth_logo.png}  
\end{minipage} 
\begin{minipage}{0.2\textwidth}
\includegraphics[width=4.3cm]{Figures/_Frontpage/ZHAW_Logo.png}  
\end{minipage}
\end{center}
%----------------------------------------------------------------------------------------

%\vfill % Fill the rest of the page with whitespace
\restoregeometry
\end{titlepage}


\begin{abstract}
Your abstract.
\end{abstract}

\thispagestyle{empty}
\tableofcontents
\newpage


\setcounter{page}{1}


\section{Introduction}
\nomenclature[A]{CFD}{Computational Fluid Dynamics}
%A cite: \cite{my_cool_citekey}
%A cite2: \cite{my_cool_citekey2}
\section{Overview of Redox Flow Batteries}

\subsection{Principles of operation}
\subsection{Battery Types}
\subsection{Materials}
\subsection{Performance}
\subsection{State of the Art}
\subsection{Challenges}

\section{Model Formulation}
\subsection{Theoretical Background}
\subsubsection{Electrochemistry}
\subsubsection{...}
\subsection{Parametrization}
\subsection{Numerical Implementation}

\section{Results and Discussion}
\subsection{Modeling Results}
\subsection{Validation and Discussion}
\subsection{Future Development}


%APPENDICES:
%\cleardoublepage
%\phantomsection
%\addcontentsline{toc}{section}{Appendices}
%\appendix

%\section{Matlab Code}\label{App:AppendixA}
%	Text of Appendix A

%NOMENCLATURE:

\newpage
\scriptsize
\printnomenclature

% WARNING! to create nomenclature run this set of commands in cmd.exe (earlier: cd command):

%pdflatex <documentname>.tex
%makeindex <documentname>.nlo -s nomencl.ist -o <documentname>.nls
%pdflatex <documentname>.tex

%or use user-defined user command for build:
%first build normally
%build with 1: Make Nomenclature (Alt+Shift+F1)
%build normally once again

% -------------------------------------------------------------------------------
% BIBLIOGRAPHY, uncomment these three:
\newpage
\bibliographystyle{abbrv}
\bibliography{JWMasterThesis} %name of .bib file
% Visit http://truben.no/latex/bibtex/# for manual bibtex entry

%in text use \cite
% -------------------------------------------------------------------------------


\end{document}
              